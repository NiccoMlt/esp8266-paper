\section{Implementazione}\label{sec:implementazione}

Esporre i principali problemi affrontati durante l'effettiva realizzazione delle componenti hardware/software e illustrare le soluzioni implementative adottate. Se l'elaborato ha previsto l'utilizzo di tecnologie già disponibili sul mercato, discuterne brevemente le caratteristiche e motivarne l'adozione rispetto ad altre soluzioni assimilabili.\\

\textbf{NOTA: in questa sezione devono essere riportate esclusivamente le porzioni di codice ritenute particolarmente significative. Il codice sorgente nella sua interezza, opportunamente commentato, deve essere consegnato separatamente dalla relazione in un archivio compresso.}\\


Vincoli circa la lunghezza della sezione (escluse didascalie, tabelle, testo nelle immagini, schemi):

\vspace{1cm}
\begin{tabular}{l|rr}
 & Numero minimo di battute & Numero massimo di battute \\
 \hline
 1 componente & 5000 & 11000 \\
 2 componenti & 8000 & 16000 \\
 3 componenti & 10000 & 21000 \\
 \hline
\end{tabular}

\subsection{Tecnologie Utilizzate}

Di seguito sono riportate in un elenco le tecnologie più rilevanti utilizzate per l'implementazione del progetto.

\begin{description}
  \item[Typescript \& Angular]
    Per la realizzazione dei frontend web del server cloud e del server edge, è stato impiegato il framework Angular con il linguaggio consigliato Typescript.
    \textbf{Typescript} è un Super-set di JavaScript ES6 e, come tale, è un linguaggio di scripting orientato agli oggetti e agli eventi
    \todo[inline]{continua}
\end{description}

\subsection{Backend cloud}

\subsection{Nodi edge}

\subsection{Sensori IoT}
