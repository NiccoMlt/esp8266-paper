\section{Introduzione}

% Esporre l’obiettivo del progetto dandone una visione complessiva.

% Devono essere illustrate: le caratteristiche salienti del progetto;
% deve essere chiara la distinzione tra le tecnologie usate/assemblate durante lo svolgimento dell'elaborato e il contributo tecnologico/scientifico effettivamente apportato dal gruppo.

% Vincoli circa la lunghezza della sezione (escluse didascalie, tabelle, testo nelle immagini, schemi):
% Numero minimo di battute per 2 componenti: 2500
% Numero massimo di battute per 2 componenti: 4500

Gli ambienti \textit{indoor} pubblicamente accessibili possono avere gradi di affollamento anche molto differenti al variare dell'orario della giornata e del periodo della settimana o dell'anno.
La possibilità di avere sotto controllo una stima della quantità di persone in una data zone ad un certo momento è senza dubbio un asset importante, applicabile in diversi domini applicativi;
ad esempio:

\begin{itemize}
  \item in un centro commerciale, sapere che in un dato periodo e/o in una data zona vi un maggior passaggio di persone può essere utile per differenziare le pubblicità su schermi pubblicitari in esposizione;
  \item in un campus universitario, avere maggiori dettagli sui momenti di maggior affollamento delle aule studio permette un uso migliore del personale e una migliore esperienza per lo studente.
\end{itemize}

Nell'ambito della raccolta di dati di questo tipo, le strategie utilizzabili sono numerose, ciascuna con differenti pregi e difetti.

In questo progetto si è scelto di sfruttare i pacchetti beacon inviati periodicamente in chiaro dai dispositivi WiFi per ottenere una stima del numero di persone presenti in una certa zona.
Questa tecnica è già stata presa in considerazione in letteratura, utilizzando diverse tipologie di ricevitori WiFi;
per questo progetto è stato realizzato un sistema di dispositivi IoT a bassissimo consumo che dialogano tramite protocolli standard e open-source tra loro e con un server in cloud per tenere traccia delle informazioni sui device dotati di WiFi all'interno delle zone monitorate.
Per il rispetto delle norme sulla privacy rispetto a questo tipo di analisi, le informazioni in merito ai dispositivi sono anonimizzate prima di essere memorizzate.
