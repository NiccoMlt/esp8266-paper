\section{Analisi di deployment su larga scala}

% In questa sezione va discussa, eventualmente con l'ausilio di opportuni diagrammi (componenti, deployment), l'evoluzione del progetto presentato immaginando che venga adottato su larga scala.
% I dettagli qui esposti devono quindi astrarre dalle specifiche dell'elaborato qualora l'implementazione sia stata focalizzata su uno scenario isolato.

% A titolo d’esempio, qualora applicabile, devono essere evidenziate le criticità che si potrebbero incontrare
% devono essere proposte soluzioni tipiche in contesti di cloud architecture per garantire un'adeguata resilienza, in termini di availability e scalability del sistema.

% Vincoli circa la lunghezza della sezione (escluse didascalie, tabelle, testo nelle immagini, schemi):
% Numero minimo di battute per 2 componenti: 6000
% Numero massimo di battute per 2 componenti: 9000

\subsection{Realizzazione reale}

Il progetto, per come è stato realizzato, costituisce un prototipo funzionante e su piccola scala di un sistema di stima di affollamento in luoghi chiusi tramite WiFi.

Per essere realmente dispiegabile, sarebbe necessario tenere in considerazione scelte hardware e software differenti, che non sono state adottate per limitazione di budget.

\subsubsection{Sensori IoT}\label{subsub:deploy:real:iot}

Per la realizzazione del prototipo di questo progetto è stato utilizzato come sensore un modulo \texttt{Espressif Systems ESP8266}\footnote{Datasheet ESP8266: \url{https://www.espressif.com/sites/default/files/documentation/0a-esp8266ex_datasheet_en.pdf}} fornito in un development kit chiamato \texttt{NodeMCU}.
Tale PCB include un modulo \texttt{ESP-12};
esso risulta adeguato per il caso di studio del prototipo, ma risulta lento nella scansione delle frequenze e le performance potrebbero peggiorare in ambienti più affollati.

Per la versione finale, l'impiego di un modulo \texttt{ESP-32}\footnote{Datasheet ESP32: \url{https://www.espressif.com/sites/default/files/documentation/esp32_datasheet_en.pdf}}
del medesimo produttore vanta un processore più veloce, un miglior ricevitore WiFi e il supporto al Bluetooth:
\begin{itemize}
  \item
    il maggior clock e il miglior ricevitore potrebbero permettere una scansione più veloce;
  \item
    il ricevitore Bluetooth potrebbe integrare il tracking anche dei pacchetti scambiati su questo protocollo per migliorare le misurazioni
    (in modo simile a quanto fatto da Waitz nel caso analizzato in \Cref{subsec:soa:waitz}).
\end{itemize}

Per aumentare ulteriormente la velocità di scansione, si potrebbero inserire un maggior numero di moduli per stanza, ciascuno impiegato solo per la scansione di un sottoinsieme delle frequenze WiFi.

\subsubsection{Connessione tra Fog e cloud}

Il server Fog implementato su \texttt{Raspberry Pi 3B} dipende molto dalla comunicazione con il cloud;
potrebbe essere utile collegare una scheda di rete LTE/4G/3G per fornire una connessione di rete di backup.
Inoltre, l'aggiunta di un DB locale potrebbe permettere un caching dei dati in caso di caduta della connessione fino alla risoluzione del problema.

\subsection{Deploy su larga scala}

Ragionando su un possibile deploy su larga scala del sistema, si evidenziano subito dei punti a favore e invece delle problematiche su cui riflettere.

Innanzitutto, il progetto è pensato per supportare una scalabilità sia orizzontale che verticale.
Partendo dal client web, pensando ad un numero di utenti elevato i problemi principali sono:

\begin{description}
  \item[Banda in ingresso e risorse]
    L'interfaccia web è una SPA realizzata in Angular;
    grazie a questa tecnologia, la gestione del routing, dell'ottenimento dei dati e della rappresentazione di questi ultimi è gestita client-side,
    rendendo i compiti del web server meno onerosi.

    Il server web locale potrebbe effettivamente essere limitato dalla velocità delle schede di rete del Raspberry Pi 3B,
    il quale deve comunicare con il cloud per risolvere i dati dal database.
    L'interfaccia web dispiegata in cloud non dovrebbe essere limitata dalle risorse attualmente fornite, ma esse possono comunque essere estese \emph{on demand} in caso di necessità.

  \item[Numerosi accessi al DB]
    Come accennato nel punto precedente, il database è acceduto spesso, dunque potrebbe essere necessario l'upgrade a un piano premium (\emph{Flame} o \emph{Blaze}) per poter gestire le richieste.
\end{description}

Inoltre, il sistema molto probabilmente vedrebbe un aumento della quantità di server Fog e sensori.
Infatti, come già detto nei punti precedenti, il poter aggiungere e rimuovere stanze ed edifici in modo semplice è una caratteristica del sistema.

Nonostante ciò sia dunque previsto, si potrebbero presentare alcune problematiche da gestire:
\begin{description}
  \item[Aumento del numero di sensori rispetto ai server Fog]
    Per quanto molto difficile per motivi di copertura del segnale WiFi, è possibile che vi sia un elevato numero di sensori connessi al medesimo server Fog che gestisce l'edificio.
    Poiché nel prototipo attuale ciascun ESP8266 si collega tramite WiFi alla rete locale generata dal Raspberry Pi usando una \texttt{netmask} di \texttt{255.255.255.0}, il numero di IP assegnabili è 255.

    Nel caso si decida di aggiungere più sensori per stanza, come ipotizzato nella \Cref{subsub:deploy:real:iot}, questo aspetto potrebbe dover essere tenuto in considerazione.
  \item[Aggiunta di server Fog]
    Come specificato nella \Cref{subsub:openapi}, la comunicazione tra cloud ed Fog è autenticata tramite token JWT\@.
    L'API cloud che lo genera richiede a sua volta un'autenticazione tramite nome zona e chiave, che devono essere configurate tramite file di configurazione secondo le best practice di Vert.x.
\end{description}

Da quanto si può evincere, gestendo le situazioni di carico estremo come evidenziato sopra, il sistema ha mantiene le caratteristiche che lo rendono \emph{scalabile}.
Le comunicazioni sono \emph{stateless} e sia server web che database si basano su tecnologie moderne e sono scalabili all'occorrenza, rimanendo responsivi, leggeri e dinamici.
