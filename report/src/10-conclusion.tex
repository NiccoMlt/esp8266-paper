\section{Conclusioni}

La realizzazione del progetto è stato un percorso stimolante e a tratti complesso. Ci ha permesso di lavorare come team e d’imparare varie tematiche affini all’ambito dell’IoT e delle Smart Cities.
L’esserci prefissati un obbiettivo abbastanza ambizioso ci ha portato a scontrarci con problematiche e difficoltà sia sul punto delle scelte architetturali che implementative. Abbiamo quindi dovuto in primo luogo accrescere le nostre competenze in materia, per poi prendere le scelte che abbiamo ritenuto più opportune.
Il poterci confrontare con realtà simili già esistenti ci ha permesso di definire gli obbiettivi in maniera più chiara e semplice.
Abbiamo comunque cercato di apportare la nostra impronta e utilizzare una chiave di lettura differente e che ci appartenesse. Ci riteniamo complessivamente abbastanza soddisfatti del nostro prototipo, che in quanto tale non è ancora un oggetto realmente utilizzabile nel mondo reale ma un buon modello per concretizzare l'idea alla base del progetto.

Come possibili miglioramenti e sviluppi futuri si potrebbe pensare di utilizzare un sensore più potente, o impiegarne di vari per migliorare sensibilmente la raccolta dati e aumentare la precisione delle funzionalitài.

Per una reale concretizzazione bisogna tenere conto anche degli aspetti di sicurezza, i quali, sono stati ragionati ma non completamente implementati a causa di un eccessivo aumento di complessità del progetto.
