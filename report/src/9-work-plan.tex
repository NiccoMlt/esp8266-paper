\section{Piano di lavoro}
\todo{Sistemare float della tabella}
\begin{table}[]
  \begin{tabular}{|c|l|l|c|}
    \hline
    \textbf{Priority} & \multicolumn{1}{c|}{\textbf{Type}} & \multicolumn{1}{c|}{\textbf{Title}} & \textbf{Points} \\ \hline
    1 & Setup & Repository Setup & 1 \\ \hline
    2 & Setup & Project Setup & 2 \\ \hline
    3 & Investigation & Crowd Tracking state of art investigation & 3 \\ \hline
    4 & Investigation & ESP sensor technology state of art investigation & 3 \\ \hline
    5 & Investigation & Best Cloud Platform choice & 1 \\ \hline
    6 & Setup & Raspberry Access Point mode configuration & 2 \\ \hline
    7 & Setup & Docker Host installation on Raspberry & 1 \\ \hline
    8 & Setup & Execute Mosquitto on Docker container & 1 \\ \hline
    9 & Setup & Configure a simple container with Vert.x Java & 1 \\ \hline
    10 & Setup & Configure Google Cloud Platform & 2 \\ \hline
    11 & Setup & Configure ESP8266 & 1 \\ \hline
    12 & Implementation & ESP8266 code library adaptation & 4 \\ \hline
    13 & Implementation & MQTT publish/subsribe interface implementation & 4 \\ \hline
    14 & Implementation & Cloud FireStore Database interface implementation & 10 \\ \hline
    15 & Implementation & API Edge implementation & 6 \\ \hline
    16 & Implementation & API Cloud implementation & 6 \\ \hline
    17 & Implementation & Web GUI edge-side implementation & 15 \\ \hline
    18 & Implementation & Web GUI cloud-side implementation & 20 \\ \hline
    19 & Implementation & Final integration testings & 8 \\ \hline
  \end{tabular}
\end{table}




Per organizzare il lavoro in team nello sviluppo del progetto è stata scelto di utilizzare
una metodologia agile organizzando le ore lavorative secondo issues prestabilite, che rappresentassero
le principali problematiche da affrontare. Il progetto è stato tracciato e mantenuto interamente su GitLab
su un repository condiviso tra i membri, in questo modo è stato possibile
lavorare in maggiore autonomia. Per la gestione delle issues è stata scelta l'interfaccia dedicata di GitLab stesso.
Circa a cadenza settimanale si è effettuato un confronto del lavoro svolto da entrambi i membri del team,
progettando e pensando ai successivi step  necessari per portare a termine il progetto.
Questa gestione è risultata più che adeguata per la
realizzazione del progetto, permettendo di procedere alla sua realizzazione senza
grossi problemi grazie ad obiettivi chiari e ad una buona suddivisione delle mansioni.

\subsection{Niccolò Maltoni}

Inserire circa mezza pagina in cui riassumere cosa è stato fatto

\subsection{Luca Semprini}

Mi sono occupato inizialmente del setup completo del Raspberry: ho configurato la modalità Access Point, installato il Docker Host ed ho inizializzato un container su cui ho eseguito Mosquitto.
Oltre alla configurazione del Raspberry, ho curato la realizzazione dell'interfaccia web nella sua interezza: sia per la parte Cloud, che per la parte Fog; ho implementato i rispettivi grafici e tabelle e la ricezione e gestione dei dati attraverso le REST api.
Infine, nella fase di testing finale ho contribuito alla risoluzione di alcuni bug, in collaborazione con il collega Maltoni.
