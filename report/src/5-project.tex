\section{Progettazione}

% Devono essere esposte le scelte progettuali operate nelle varie fasi di sviluppo dell'elaborato.

% In questa sezione devono essere documentati gli schemi di progetto relativamente all'architettura complessiva del sistema e alle sue componenti di rilievo.
% Per le componenti software si può ricorrere ad esempio a diagrammi delle classi, di sequenza, stato, attività.
% Per le componenti hardware è possibile includere opportuni schemi in grado di descrivere l'architettura fisica adottata.

% Vincoli circa la lunghezza della sezione (escluse didascalie, tabelle, testo nelle immagini, schemi):
% Numero minimo di battute per 2 componenti: 12000
% Numero massimo di battute per 2 componenti: 21000

In questa sezione viene presentata l'architettura generale del sistema, partendo da come è stata derivata a partire dai requisiti raccolti,
seguendo un approccio top-down: si affronterà quindi prima la progettazione dell'architettura ad alto livello
e in un secondo momentole effettive implementazioni e architetture di dettaglio.

\subsection{Architettura generale del sistema}

\subsection{Interazioni tra gli elementi del sistema}

\subsection{Hardware necessario}
